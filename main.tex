\documentclass{rapportCS}
\usepackage{lipsum}

% Charger les commandes custom si besoin

%%%%%%%%%%%%%%%%%%%%%% LES PACKAGES
% TODO : Delete .cls and refactoring

%%%%%%%%%%%%%%%%%%%%%% MISE EN FORME
\input{config/formatting}

%%%%%%%%%%%%%%%%%%%%%%% LA PALETTE DE COULEURS
\input{config/palette}

%%%%%%%%%%%%%%%%%%%%%%% LES COMMANDES 
%symboles pratiques
\newcommand{\flch}{\item[$\rightarrow$]}
\newcommand{\dc}{{\usebeamercolor[fg]{structure}$\hookrightarrow$}}
\newcommand{\ok}{\textcolor{vert}{\checkmark}}
\newcommand{\point}{{\usebeamercolor[fg]{structure}$\bullet\enskip$}}
\newcommand{\Point}{{\usebeamercolor[fg]{structure}$\bullet\enskip$}}

%styles
\newcommand{\couleur}[1]{{\usebeamercolor[fg]{structure}#1}}
\newcommand{\important}[1]{\couleur{\textbf{#1}}}
\newcommand{\remarque}[1]{\textit{\textrm{#1}}}

%pour le template
\newcommand{\lin}[1]{\mintinline{latex}{#1}}



\title{Rapport CentraleSupelec - Template}

\begin{document}

%----------- Informations du rapport ---------

\logoentreprise{logos/logoGraphene.png}
\titre{Mon Titre} % FIXME: Définir le titre
\mention{Cursus ingénieur}
\trigrammemention{ING2}
\master{Génie des Systèmes d'Information}
\filiere{Seconde année}

%----------- Initialisation -------------------
        
\fairemarges
\fairepagedegarde

%------------ Table des matières ----------------
\newpage
\bgroup
\pagestyle{empty}
\makenomenclature
\pagestyle{fancy}
\fancyheadoffset{0.5cm}
\rhead{\nouppercase{\leftmark}}
\cfoot{\textbf{\titre}}
\rfoot{\thepage}
\lfoot{\trigrammemention}
\tabledematieres 
\clearpage
\egroup 

%############## Corps du rapport ##############

\section{Introduction}
\lipsum[1-2]

% Pour citer: \cite{clef_biblio}

\section{Développement}
\lipsum[3-5]

%%%%%%%%%%%%%%%%%%%%%%%%%%%%%
%       Bibliographie       %
%%%%%%%%%%%%%%%%%%%%%%%%%%%%%
\newpage
% Décommenter si vous utilisez la biblio:
% \bibliographystyle{plain}
% \bibliography{biblio}

\appendix
% \include{src/annexe}

\end{document}
